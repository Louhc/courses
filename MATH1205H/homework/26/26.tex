\documentclass[12pt, a4paper, oneside]{article}

\usepackage[top=2cm, bottom=1.5cm, left=1.8cm, right=1.5cm]{geometry} % 页边距
\usepackage{upgreek, graphicx, bm, slashed, amsmath, amssymb, lmodern, simplewick, color, fancyhdr}
%\spaceskip=0.2em % 调节空格大小
\title{Fall 2022 MATH1205H Homework XXVI}
\author{Lou Hancheng \quad louhancheng@sjtu.edu.cn}
\date{\today}
\pagestyle{fancy}
\fancyhead{}
\fancyhead[L]{MATH1205H}
\fancyhead[R]{Lou Hancheng (522010910160)}

\begin{document}
    \maketitle
    \section*{Exercise 1.}
        No.Because we can change the order of $\sigma$ sequence.
    \section*{Exercise 2.}
        $$
            Ax=\lambda x
        $$
        $$
            \left\{\begin{aligned}
                & 2x + y &= \lambda x\\
                & 4x + 2y &= \lambda y
            \end{aligned}\right.
        $$
        $$
            \left\{\begin{aligned}
                & \lambda_1 &= 4\\
                & \lambda_2 &= 0
            \end{aligned}\right.
        $$
        $$
            A^TA=\begin{bmatrix}
                20 & 10\\
                10 & 5
            \end{bmatrix}
        $$
        $$
            \left\{\begin{aligned}
                & \lambda_1' &= 25\\
                & \lambda_2' &= 0
            \end{aligned}\right.
        $$
        $$
            \left\{\begin{aligned}
                & \sigma_1 &= 5\\
                & \sigma_2 &= 0
            \end{aligned}\right.
        $$
    \section*{Exercise 3.}
        $$
            A^TA=\begin{bmatrix}
                2 & 1\\
                1 & 1
            \end{bmatrix}
        $$
        $$
            \left\{\begin{aligned}
                & 2x + y &= \lambda x\\
                & x + y &= \lambda y
            \end{aligned}\right.
        $$
        $$
            \left\{\begin{aligned}
                \lambda_1 &= \frac{3-\sqrt5}{2} & v_1 &= \sqrt{\frac{2}{5+\sqrt{5}}}\begin{bmatrix}
                    1 & -\frac{1+\sqrt{5}}{2}
                \end{bmatrix}^T & u_1 &= \frac{Av_1}{\sqrt{\lambda_1}} &=\sqrt{\frac{2}{5+\sqrt{5}}}\begin{bmatrix}
                    -1 & \frac{1+\sqrt{5}}{2}
                \end{bmatrix}\\
                \lambda_2 &= \frac{3+\sqrt5}{2} & v_2 &= \sqrt{\frac{2}{5-\sqrt{5}}}\begin{bmatrix}
                    1 & \frac{\sqrt{5}-1}{2}
                \end{bmatrix}^T & u_2 &= \frac{Av_2}{\sqrt{\lambda_2}} &=\sqrt{\frac{2}{5-\sqrt{5}}}\begin{bmatrix}
                    1 & \frac{\sqrt{5}-1}{2}
                \end{bmatrix}
            \end{aligned}\right.
        $$
        $$
            \left\{\begin{aligned}
                \sigma_1 &= \frac{\sqrt{5} - 1}{2}\\
                \sigma_2 &= \frac{\sqrt{5} + 1}{2}
            \end{aligned}\right.
        $$
        $$
            \begin{aligned}
                A&=\begin{bmatrix}
                    \sqrt{\frac{2}{5+\sqrt{5}}} & \sqrt{\frac{2}{5-\sqrt{5}}}\\
                    -\sqrt{\frac{2}{5+\sqrt{5}}}\frac{1+\sqrt{5}}{2} & \sqrt{\frac{2}{5-\sqrt{5}}}\frac{\sqrt{5}-1}{2}
                \end{bmatrix}\begin{bmatrix}
                    \frac{\sqrt{5} - 1}{2} & 0\\
                    0 & \frac{\sqrt{5} + 1}{2}
                \end{bmatrix}\begin{bmatrix}
                    -\sqrt{\frac{2}{5+\sqrt{5}}} & \sqrt{\frac{2}{5+\sqrt{5}}}\frac{\sqrt{5}+1}{2}\\
                    \sqrt{\frac{2}{5-\sqrt{5}}} & \sqrt{\frac{2}{5-\sqrt{5}}}\frac{\sqrt{5}-1}{2}
                \end{bmatrix}\\&=\begin{bmatrix}
                    \sqrt{\frac{2}{5+\sqrt{5}}} & \sqrt{\frac{2}{5-\sqrt{5}}}\\
                    -\sqrt{\frac{2}{5-\sqrt{5}}} & \sqrt{\frac{2}{5+\sqrt{5}}}
                \end{bmatrix}\begin{bmatrix}
                    \frac{\sqrt{5} - 1}{2} & 0\\
                    0 & \frac{\sqrt{5} + 1}{2}
                \end{bmatrix}\begin{bmatrix}
                    -\sqrt{\frac{2}{5+\sqrt{5}}} & \sqrt{\frac{2}{5-\sqrt{5}}}\\
                    \sqrt{\frac{2}{5-\sqrt{5}}} & \sqrt{\frac{2}{5+\sqrt{5}}}
                \end{bmatrix}
            \end{aligned}
        $$
    \section*{Exercise 4.}
        $$
            A=\sum_{i\in[n]} \sigma_iu_iv_i^T
        $$
        Let $\sigma_i=\frac{1}{\lVert v_i\rVert^2}(\forall i\in [n])$
        $$
            Av_j=\sum_{i\in[n]} \sigma_iu_i(v_i^Tv_j)
        $$
        $v_1, ..., v_n$ are orthonormal, so
        $$
            v_i^Tv_j =\left\{\begin{aligned}
                & \lVert v_i\rVert^2  & i=j\\
                & 0 & i\neq j
            \end{aligned}\right.            
        $$
        Therefore,
        $$
            Av_i=u_i (\forall i\in [n])
        $$
        $$
            A=U\begin{bmatrix}
                \frac{1}{\lVert v_1\rVert^2} & 0 & \cdots & 0\\
                0 & \frac{1}{\lVert v_2\rVert^2} & \cdots & 0\\
                \vdots & \vdots & \ddots & \vdots\\
                0 & 0 & \cdots & \frac{1}{\lVert v_n\rVert^2}
            \end{bmatrix}V^T
        $$
    \section*{Exercise 5.}
        \subsection*{sufficiency}
            $$
                \forall \lambda\text{ is an eigenvalue with a corresponding eigenvector }v:Sv=\lambda v, \lambda v^Tv=v^TSv
            $$
            $$
                \lambda = \frac{v^TSv}{v^Tv}\geq 0
            $$
            So it's semidefinite.
        \subsection*{neccessity}
            $$
                S=\lambda_1v_1v_1^T+\cdots+\lambda_nv_nv_n^T (\forall i\in [n]:\lambda_i\geq 0)
            $$
            Then
            $$
                x^TSx=\sum_{i\in[n]}\lambda_ix^Tv_iv_i^Tx=\sum_{i\in[n]}\lambda_i(v_i^Tx)^2\geq 0
            $$
\end{document}