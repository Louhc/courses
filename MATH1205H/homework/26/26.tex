\documentclass[12pt, a4paper, oneside]{article}

\usepackage[top=2cm, bottom=1.5cm, left=1.8cm, right=1.5cm]{geometry} % 页边距
\usepackage{upgreek, graphicx, bm, slashed, amsmath, amssymb, lmodern, simplewick, color, fancyhdr}
%\spaceskip=0.2em % 调节空格大小
\title{Fall 2022 MATH1205H Homework XXVI}
\author{Lou Hancheng \quad louhancheng@sjtu.edu.cn}
\date{\today}
\pagestyle{fancy}
\fancyhead{}
\fancyhead[L]{MATH1205H}
\fancyhead[R]{Lou Hancheng (522010910160)}

\begin{document}
    \maketitle
    \section*{Exercise 1.}
        No.Because we can change the order of $\sigma$ sequence.
    \section*{Exercise 2.}
        $$
            Ax=\lambda x
        $$
        $$
            \left\{\begin{aligned}
                & 2x + y &= \lambda x\\
                & 4x + 2y &= \lambda y
            \end{aligned}\right.
        $$
        $$
            \left\{\begin{aligned}
                & \lambda_1 &= 4\\
                & \lambda_2 &= 0
            \end{aligned}\right.
        $$
        $$
            A^TA=\begin{bmatrix}
                20 & 10\\
                10 & 5
            \end{bmatrix}
        $$
    \section*{Exercise 5.}
        \subsection*{sufficiency}
            $$
                \forall \lambda\text{ is an eigenvalue with a corresponding eigenvector }v:Sv=\lambda v, \lambda v^Tv=v^TSv
            $$
            $$
                \lambda = \frac{v^TSv}{v^Tv}\geq 0
            $$
            So it's semidefinite.
        \subsection*{neccessity}
            $$
                S=\lambda_1v_1v_1^T+\cdots+\lambda_nv_nv_n^T (\forall i\in [n]:\lambda_i\geq 0)
            $$
            Then
            $$
                x^TSx=\sum_{i\in[n]}\lambda_ix^Tv_iv_i^Tx=\sum_{i\in[n]}\lambda_i(v_i^Tx)^2\geq 0
            $$
\end{document}