\documentclass[12pt, a4paper, oneside]{article}

\usepackage[top=2cm, bottom=1.5cm, left=1.8cm, right=1.5cm]{geometry} % 页边距
\usepackage{upgreek, graphicx, bm, slashed, amsmath, amssymb, lmodern, simplewick, color, fancyhdr}
%\spaceskip=0.2em % 调节空格大小
\title{Fall 2022 MATH1205H Homework XXV}
\author{Lou Hancheng \quad louhancheng@sjtu.edu.cn}
\date{\today}
\pagestyle{fancy}
\fancyhead{}
\fancyhead[L]{MATH1205H}
\fancyhead[R]{Lou Hancheng (522010910160)}

\begin{document}
    \maketitle
    \section*{Exercise 1.}
        If $L$ is not the zero function,
        $$
            \exists v\in V:L(v)\neq 0
        $$
        So
        $$
            \forall x\in \mathbb{R}:x=\frac{x}{L(v)}L(v)=L(\frac{x}{L(v)}v)
        $$
        Therefore, $L$ is surjective.
    \section*{Exercise 2.}
        $$
            \forall v\in V : T'(L)(v)=L(I(v))=L(v)
        $$
        Therefore, $\forall L\in V':T'(L)=L$
    \section*{Exercise 3.}
        Choose an arbitrary basis $\tilde{v}_1,...,\tilde{v}_n$ and the corresponding dual basis $\tilde{L}_1,...,\tilde{L}_n$.
        $$
            \exists M=\begin{bmatrix}
                x_1 & \cdots & x_n
            \end{bmatrix}\in \mathbb{R}^{n\times n}:\begin{bmatrix}
                \tilde{L}_1 & \cdots & \tilde{L}_n
            \end{bmatrix}M=\begin{bmatrix}
                L_1 & \cdots & L_n
            \end{bmatrix}
        $$
        $$
            L_i=\begin{bmatrix}
                \tilde{L}_1 & \cdots & \tilde{L}_n
            \end{bmatrix}x_i
        $$
        Suppose
        $$
            (M^{-1})^T=\begin{bmatrix}
                y_1 & \cdots & y_n
            \end{bmatrix}
        $$
        So
        $$
            x_i\cdot y_j=\left\{\begin{aligned}
                & 0 & i\neq j\\
                & 1 & i = j
            \end{aligned}\right.
        $$
        Let
        $$
            v_i=\begin{bmatrix}
                \tilde{v}_1 & \cdots & \tilde{v}_n
            \end{bmatrix}y_i
        $$
        Then
        $$
            L_i(v_i)=x_i\cdot y_i=\left\{\begin{aligned}
                & 0 & i\neq j\\
                & 1 & i = j
            \end{aligned}\right.
        $$
        So the corresponding dual basis for $v_1, ..., v_n$ is precisely $L_1, ..., L_n$
    \section*{Exercise 4.}
        \subsection*{$T=0\Rightarrow T'=0$}
            $$
                \forall v\in V:T(v)=0
            $$
            Therefore
            $$
                \forall L\in W',\forall v\in V:T'(L)(v)=L(Tv)=L(0)=0
            $$
            $$
                \forall L\in W':T'(L)=0
            $$
            So $T'=0$
        \subsection*{$T'=0\Rightarrow T=0$}
            $$
                \forall L\in W', \forall v\in V:L(Tv)=T'(L)(v)=0
            $$
            If $T\neq 0$, i.e., $\exists v_0\in V:Tv_0\neq 0$
            $$
                T'((Tv_0)')(v_0)=(Tv_0)'(Tv_0)=1
            $$
            which contradicts with above.So $T=0$
    \section*{Exercise 5.}
        \subsection*{sufficiency}
            If
            $$
                A=uv^T=\begin{bmatrix}
                    u_1v_1 & u_1v_2 & \cdots & u_1v_n\\
                    u_2v_1 & u_2v_2 & \cdots & u_2v_n\\
                    \vdots & \vdots & \ddots & \vdots\\
                    u_mv_1 & u_mv_2 & \cdots & u_mv_n
                \end{bmatrix}=\begin{bmatrix}
                    a_1 & a_2 & \cdots & a_n
                \end{bmatrix}
            $$
            Without loss of generosity, we suppose that $v_1\neq 0$.So
            $$
                \forall i\geq 2: a_i=\frac{v_i}{v_1}a_1\in\text{span}\{a_1\}
            $$
            So $a_1$ is a basis for $C(A)$ and $\text{rank}(A)=1$
        \subsection*{neccessity}
            $\text{rank}(A)=1$ and suppose $u$ is a basis of $C(A)$, then
            $$
                A=\begin{bmatrix}
                    v_1u & v_2u & \cdots & v_mu
                \end{bmatrix} (\prod_{i\in[m]}v_i\neq 0)
            $$
            So
            $$
                A=u\begin{bmatrix}
                    v_1 & v_2 & \cdots & v_m
                \end{bmatrix}=uv^T
            $$
\end{document}