\documentclass[12pt, a4paper, oneside]{ctexart}

\usepackage[top=2cm, bottom=1.5cm, left=1.8cm, right=1.5cm]{geometry} % 页边距
\usepackage{upgreek, graphicx, bm, slashed, amsmath, amssymb, lmodern, simplewick, color, fancyhdr}
%\spaceskip=0.2em % 调节空格大小
\title{Fall 2022 MATH1607H Homework 5}
\author{Lou Hancheng \quad louhancheng@sjtu.edu.cn}
\date{\today}
\pagestyle{fancy}
\fancyhead{}
\fancyhead[L]{MATH1607H}
\fancyhead[R]{Lou Hancheng (522010910160)}

\begin{document}
    \maketitle
    \section*{第7章第5节}
        \subsection*{4.}
            $$
                \begin{aligned}
                    m&=\int_{0}^{2\pi} \rho z\sqrt{(x')^2+(y')^2+(z')^2} \,dt\\
                    &=\int_{0}^{2\pi} (\rho b\sqrt{a^2+b^2})t \,dt\\
                    &=2\pi^2\rho b\sqrt{a^2+b^2}
                \end{aligned}
            $$
        \subsection*{7.}
            $$
                \begin{aligned}
                    E&=\int_{-r}^{r} \frac12\omega^2(r^2-x^2)\cdot2\pi\rho\sqrt{r^2-x^2}\sqrt{1+(y')^2}\,dx\\
                    &=\pi\rho\omega^2\int_{0}^{\pi}(r^2-x^2)r \,dx\\
                    &=\frac{4\pi}{3}\rho\omega^2r^4
                \end{aligned}
            $$
        \subsection*{9.}
            $$
                \begin{aligned}
                    W&=-\int_{1}^{T}kv^3\,dt\\
                    &=-k\int_{1}^{T}(9t^2-1)^3\,dt\\
                    &=k(-\frac{729}{7}T^7+\frac{243}{5}T^5-9T^3+T+\frac{2224}{35})
                \end{aligned}
            $$
        \subsection*{10.}
            $$
                \begin{aligned}
                    dh&=\frac{(0.01)^2\pi v \,dt}{1^2\pi}\\
                    &=\frac{3}{5}\times 10^{-4}\sqrt{2gh}\,dt
                \end{aligned}
            $$
            $$
                \begin{aligned}
                    t=\frac{1}{3\times 10^{-5}}\sqrt{\frac{H}{2g}}\approx 10541
                \end{aligned}
            $$
        \subsection*{14.}
            $$
                dp=k(p_{\max}-p(t))dt, p(t_0)=p_0
            $$
            解得
            $$
                p(t)=p_{\max}-(p_{\max}-p_0)e^{-k(t-t_0)}
            $$
    \section*{第8章第1节}
        \subsection*{1.}
            $$
                \begin{aligned}
                    \varphi(x)&=\frac1q\int_{x}^{+\infty}F\,dr\\
                    &=k\int_{x}^{+\infty}\frac{1}{r^2}\,dr\\
                    &=\frac{k}{x}
                \end{aligned}
            $$
        \subsection*{3.(1)}
            $$
                \int_{0}^{+\infty}e^{-2x}\sin 5x\,dx = (-\frac{2}{29}\sin 5x - \frac{10}{29}\cos 5x)e^{-2x}|^{+\infty}_{0}=-\frac{10}{29}
            $$
        \subsection*{3.(3)}
            $$
                \int_{-\infty}^{+\infty}\frac{1}{1+x+x^2}\,dx = \frac{2}{\sqrt{3}}\arctan{\frac{2}{\sqrt{3}}(x+\frac12)}|_{-\infty}^{+\infty}=\pi
            $$
        \subsection*{3.(6)}
            当 $p>1$
            $$
                \int_{2}^{+\infty}\frac{1}{x\ln^px}\,dx=-\frac{1}{(p-1)\ln^{p-1}x}|^{+\infty}_{2}=\frac{1}{(p-1)\ln^{p-1}2}
            $$
            当 $p<1$
            $$
                \int_{2}^{+\infty}\frac{1}{x\ln^px}\,dx=-\frac{1}{(p-1)\ln^{p-1}x}|^{+\infty}_{2}
            $$
            是发散的.
            当 $p=1$
            $$
                \int_{2}^{+\infty}\frac{1}{x\ln x}\,dx=\ln(\ln x)|^{+\infty}_{2}
            $$
            是发散的.
        \subsection*{3.(10)}
            $$
                \begin{aligned}
                    \int_{0}^{+\infty}\frac{\ln x}{1+x^2}\,dx&=\int_{0}^{1}\frac{\ln x}{1+x^2}\,dx+\int_{1}^{+\infty}\frac{\ln x}{1+x^2}\,dx\\
                    &= \int_{0}^{1}\frac{\ln x}{1+x^2}-\int_{0}^{1}\frac{\ln x}{1+x^2}\\
                    &=0
                \end{aligned}
            $$
        \subsection*{4.(2)}
            $$
                \int_{1}^{e}\frac{1}{x\sqrt{1-\ln^2x}}\,dx=\arcsin{\ln x}|_{1}^{e}=\frac{\pi}{2}
            $$
        \subsection*{4.(5)}
            $$
                \int_{-1}^{1}\frac{1}{x^3}\sin\frac{1}{x^2}\,dx=\frac12\cos\frac{1}{x^2}|_{-1}^{1}=0
            $$
        \subsection*{4.(6)}
            $$
                \int_{0}^{\frac12\pi}\frac{1}{\sqrt{\tan x}}\,dx=2\int_{0}^{+\infty}\frac{1}{x^4+1}\,dx=\frac{\pi}{2\sqrt2}
            $$
        \subsection*{5.}
            $$
                \ln\lim_{n\to\infty}\frac{\sqrt[n]{n!}}{n}=\lim_{n\to\infty}\sum_{i=1}^n\frac{\ln\frac{i}{n}}{n}=\int_{0}^1\ln x\,dx=x(\ln x - 1)|^{1}_{0}=-1
            $$
            $$
                \lim_{n\to\infty}\frac{\sqrt[n]{n!}}{n}=\frac1e
            $$
        \subsection*{6.(2)}
            $$
                \int_{0}^{\pi}x\ln\sin x\,dx=\pi\int_{0}^{\frac12\pi}\ln\sin x\,dx=-\frac{\pi^2}{2}\ln2
            $$
        \subsection*{6.(5)}
            $$
                \int_{0}^{1}\frac{\ln x}{\sqrt{1-x^2}}\,dx=\int_{0}^{\pi}x\ln\sin x\,dx=-\frac{\pi^2}{2}\ln 2
            $$
        \subsection*{7.(3)}
            $$
                (\text{cpv})\int_{\frac12}^{2}\frac{1}{x\ln x}\,dx=\int_{1}^{2}(\frac{1}{x\ln x}-\frac{1}{x\ln x})\,dx=0
            $$
        \subsection*{9.(1)}
            $$
                \begin{aligned}
                    f(x)\leq g(x)&\Rightarrow \int_{a}^{b}f(x)\,dx\leq \int_{a}^{b}g(x)\,dx \Rightarrow \lim_{b\to+\infty} \int_{a}^{b}f(x)\,dx\leq \lim_{b\to+\infty}\int_{a}^{b}g(x)\,dx\\
                    &\Rightarrow\int_{a}^{+\infty}f(x)\,dx\leq \int_{a}^{+\infty}g(x)\,dx
                \end{aligned}
            $$
            $$
                \begin{aligned}
                    \int_{a}^{c}f(x)\,dx=\int_{a}^{b}f(x)\,dx+\int_{b}^{c}f(x)\,dx&\Rightarrow \lim_{c\to+\infty}\int_{a}^{c}f(x)\,dx=\int_{a}^{b}f(x)\,dx+\lim_{c\to+\infty}\int_{b}^{c}f(x)\,dx\\
                    &\Rightarrow \int_{a}^{+\infty}f(x)\,dx=\int_{a}^{b}f(x)\,dx+\int_{b}^{+\infty}f(x)\,dx
                \end{aligned}
            $$
        \subsection*{9.(2)}
            $$
                \begin{aligned}
                    & \int_{0}^{1}\frac{1}{\sqrt{x}}\,dx=2\\
                    & \int_{0}^{1}\frac1x\,dx \text{是发散的.}
                \end{aligned}
            $$
        \subsection*{11.}
            不妨设$A\geq 0$
            $$
                \begin{aligned}
                    &\lim_{x\to+\infty}f(x)=A\Rightarrow \exists X>a,\forall x>X:f(x)>\frac{A}{2}\\
                    \Rightarrow &\int_{a}^{+\infty}f(x)\,dx=\int_{a}^{X}f(x)\,dx+\int_{X}^{+\infty}f(x)\,dx>\int_{a}^{X}f(x)\,dx+\int_{X}^{+\infty}\frac{A}{2}\,dx
                \end{aligned}
            $$
            若 $A>0$, 显然是发散的.因此若$\int_{a}^{+\infty}f(x)\,dx$是收敛的, $A=0$
        \subsection*{12.}
            $$
                \int_{a}^{+\infty}f'(x)\,dx=\lim_{x\to+\infty}f(x)-f(a)
            $$
            由于 $\int_{a}^{+\infty}f'(x)\,dx$ 收敛, $\lim_{x\to+\infty}f(x)=A$.
            由 11. 的结论, $\int_{a}^{+\infty}f'(x)\,dx=A=0$.
        \subsection*{(**)}
            $x+1\leq e^x$, 因此
            $$
                (1-x^2)^n\leq e^{-nx^2}\leq\frac{1}{(x^2 + 1)^n}
            $$
            两边求定积分
            $$
                \int_{0}^{1} (1-x^2)^n\,dx \leq \int_{0}^{1}e^{-nx^2}\,dx\leq\int_{0}^{1}\frac{1}{(x^2 + 1)^n}\,dx
            $$
            $$
                \int_{0}^{1} \sqrt{n}(1-x^2)^n\,dx \leq \int_{0}^{\sqrt{n}}e^{-x^2}\,dx\leq\int_{0}^{1}\frac{\sqrt{n}}{(x^2 + 1)^n}\,dx
            $$
            即
            $$
                \sqrt{n}\frac{(2n)!!}{(2n+1)!!}\leq \int_{0}^{\sqrt{n}}e^{-x^2}\,dx\leq \sqrt{n}\frac{(2n-3)!!}{(2n-2)!!}\frac{\pi}{2}
            $$
            即
            $$
                \sqrt{\frac{n}{2n+1}}\sqrt{(\frac{(2n)!!}{(2n-1)!!})^2\frac{1}{2n+1}}\leq \int_{0}^{\sqrt{n}}e^{-x^2}\,dx\leq \sqrt{\frac{n}{2n-1}}\sqrt{\frac{1}{(\frac{(2n-2)!!}{(2n-3)!!})^2\frac{1}{2n-1}}}\frac{\pi}{2}
            $$
            对 $n$ 求极限得
            $$
                \int_{0}^{\sqrt{n}}e^{-x^2}\,dx=\frac{\sqrt{\pi}}{2}
            $$
    \section*{第8章第2节}
        \subsection*{3.(4)}
            $$
                \frac{x^q}{1+x^p}\sim x^{q-p}
            $$
            因此当 $q-p<-1$ 时收敛,当 $q-p\geq -1$ 时发散.
        \subsection*{4.}
            显然若 $\int_{-\infty}^{+\infty}f(x)\,dx$ 收敛, $(\text{cpv})\int_{-\infty}^{+\infty}f(x)\,dx$ 也收敛.
            若 $(\text{cpv})\int_{-\infty}^{+\infty}f(x)\,dx$ 收敛,
            $$
                (\text{cpv})\int_{-\infty}^{+\infty}f(x)\,dx=\int_{0}^{+\infty}(f(x)+f(-x))\,dx
            $$
            也是收敛的.由柯西收敛原理,
            $$
                \forall \varepsilon > 0, \exists A \geq 0, \forall x_1\geq x_2\geq A: |\int_{x_1}^{x_2}f(x)\,dx+\int_{x_1}^{x_2}f(-x)\,dx|\leq \varepsilon
            $$
            由于 $\int_{x_1}^{x_2}f(x)\,dx\geq 0$, $\int_{x_1}^{x_2}f(-x)\,dx\geq 0$.
            $$
                |\int_{x_1}^{x_2}f(x)\,dx|\leq \varepsilon
            $$
            因此 $\int_{0}^{+\infty}f(x)\,dx$ 是收敛的. 同理 $\int_{-\infty}^{0}f(x)\,dx$ 也是收敛的. 因此 $\int_{-\infty}^{+\infty}f(x)\,dx$ 是收敛的.
        \subsection*{5.(3)}
            当 $p > 1$ 时,
            $$
                \int_{1}^{+\infty}\frac{|\sin x\arctan x|}{x^p}\,dx\leq \int_{1}^{+\infty}\frac{\pi}{2x^p}\,dx
            $$
            因此是绝对收敛的.\\
            当 $0<p \leq 1$ 时,
            $$
                \int_{1}^{+\infty}\frac{|\sin x\arctan x|}{x^p}\,dx=\int_{1}^{+\infty}\frac{|\sin x|}{x^p}\arctan x\,dx\geq \frac{\pi}{4}\int_{1}^{+\infty}\frac{|\sin x|}{x^p}
            $$
            是发散的.\\
            $F(A)=\int_{1}^{A}\sin x\,dx$上有界, $\frac{1}{x^p}$单调且 $\lim_{x\to+\infty}\frac{1}{x^p}=0$.由Dirichlet判别法得
            $$
                \int_{1}^{+\infty}\frac{\sin x}{x^p}\,dx
            $$
            是收敛的. 又因为 $\arctan x$ 单调有界,由Abel判别法得
            $$
                \int_{1}^{+\infty}\frac{\sin x\arctan x}{x^p}\,dx
            $$
            是收敛的.因此是条件收敛的.
        \subsection*{5.(4)}
            $$
                \int_{0}^{A}\sin(x^2)\,dx=\int_{0}^{\sqrt{A}}\frac{\sin x}{2\sqrt x}\,dx
            $$
            因为 $F(A)=\int_{0}^{A}\sin x\,dx$ 有界且 $\frac{1}{2\sqrt{x}}$单调且 $\lim_{x\to \infty}\frac{1}{2\sqrt{x}}=0$
            $$
                \int_{0}^{+\infty}\frac{\sin x}{2\sqrt{x}}\,dx
            $$
            收敛.\\
            因此
            $$
                \int_{0}^{+\infty}\sin(x^2)\,dx=\lim_{A\to\infty}\int_{0}^{A}\sin(x^2)\,dx=\lim_{A\to\infty}\int_{0}^{\sqrt{A}}\frac{\sin(x)}{2\sqrt{x}}\,dx
            $$
            是收敛的.\\
            显然
            $$
                \int_{0}^{+\infty}|\sin(x^2)|\,dx
            $$
            发散.\\
            因此
            $$
                \int_{0}^{+\infty}\sin(x^2)\,dx
            $$
            条件收敛.
        \subsection*{6. 8.2.3'}
            \subsubsection*{(1)}
                $\forall \varepsilon > 0, \exists \delta > 0, \forall A,A'\in (b-\delta, b):$
                $$
                    |\int_{A}^{A'}f(x)\,dx|<\varepsilon
                $$
                $$
                    \begin{aligned}
                        |\int_{A}^{A'}f(x)g(x)\,dx|&=|g(A)\int_{A}^{\xi}f(x)\,dx+g(A')\int_{\xi}^{A'}f(x)\,dx|\\
                        &\leq|g(A)\int_{A}^{\xi}f(x)\,dx|+|g(A')\int_{\xi}^{A'}f(x)\,dx|\\
                        &\leq 2M\varepsilon
                    \end{aligned}
                $$
                因此
                $$
                    \int_{a}^{b}f(x)g(x)\,dx
                $$
                是收敛的.
            \subsubsection*{(2)}
                $\forall \varepsilon > 0, \exists \delta > 0, \forall x\in (b-\delta, b):$
                $$
                    |g(x)|<\varepsilon
                $$
                $$
                    \begin{aligned}
                        |\int_{A}^{A'}f(x)g(x)\,dx|&=|g(A)\int_{A}^{\xi}f(x)\,dx+g(A')\int_{\xi}^{A'}f(x)\,dx|\\
                        &\leq|g(A)\int_{A}^{\xi}f(x)\,dx|+|g(A')\int_{\xi}^{A'}f(x)\,dx|\\
                        &\leq 4G\varepsilon
                    \end{aligned}
                $$
                因此
                $$
                    \int_{a}^{b}f(x)g(x)\,dx
                $$
                是收敛的.
        \subsection*{7.(5)}
            $$
                \forall t \in (0,1) : \lim_{x\to 0^+}|\ln x|^px^t=0
            $$
            因此
            $$
                \int_{0}^{\frac12}|\ln x|^p\,dx
            $$
            是收敛的.
            $$
                |\ln x|^{p}\sim (1-x)^p (x\to 1^+)
            $$
            因此当 $p>-1$
            $$
                \int_{\frac12}^{1}|\ln x|^p\,dx
            $$
            是收敛的.\\
            当 $p\leq -1$
            $$
                \int_{\frac12}^{1}|\ln x|^p\,dx
            $$
            是发散的.\\
            综上,当 $p>-1$
            $$
                \int_{0}^{1}|\ln x|^p\,dx
            $$
            是收敛的.\\
            当 $p\leq -1$
            $$
                \int_{0}^{1}|\ln x|^p\,dx
            $$
            是发散的.\\
        \subsection*{7.(6)}
            $$
                x^{p-1}(1-x)^{q-1}\sim x^{p-1} (x\to 0)
            $$
            $$
                x^{p-1}(1-x)^{q-1}\sim x^{q-1} (x\to 1)
            $$
            因此当且仅当 $p>0,q>0$ 时收敛, 其他情况发散.
        \subsection*{8.(8)}
            $$
                \lim_{x\to +\infty}\frac{1}{x^p\ln^qx}x^p=0
            $$
            因此若 $p \geq 1$
            $$
                \int^{+\infty}_{2}\frac{1}{x^p\ln^q x}\,dx
            $$
            收敛.\\
            当 $p < 1$ 时
            $$
                \exists X \forall x > X:\frac{1}{x^p\ln^qx}> \frac1x
            $$
            因此发散.
        \subsection*{9.(3)}
            $$
                \int_{0}^{+\infty}\frac{e^{\sin x}\cos x}{x^p}\,dx=\int_{0}^{1}\frac{e^{\sin x}\cos x}{x^p}+\int_{1}^{+\infty}\frac{e^{\sin x}\cos x}{x^p}
            $$
            由于
            $$
                \frac{e^{\sin x}\cos x}{x^p}\sim \frac{1}{x^p}(x\to 0^+)
            $$
            因此当 $p<1$ 时收敛, 当 $p\geq 1$ 时发散.\\
            当 $p\leq 0$ 时,
            $$
                \int_{1}^{+\infty}\frac{e^{\sin x}\cos x}{x^p}\,dx
            $$
            显然发散.
            当 $0<p<1$ 时,
            $$
                F(A)=\int_{1}^{A}e^{\sin x}\cos x\,dx
            $$
            有界,
            $$
                \lim_{x\to+\infty}\frac{1}{x^p}=0
            $$
            因此 
            $$
                \int_{1}^{+\infty}\frac{e^{\sin x}\cos x}{x^p}\,dx
            $$
            收敛.\\
            综上,当 $0<p<1$ 时
            $$
                \int_{1}^{+\infty}\frac{e^{\sin x}\cos x}{x^p}\,dx
            $$
            收敛, 其余情况发散.
        \subsection*{9.(6)}
            当 $p>1$ 时
            $$
                \frac{|\sin(x+\frac1x)|}{x^p}\leq \frac{1}{x^p}
            $$
            因此收敛.\\
            当 $0<p\leq 1$时
            $$
                \int_{1}^{+\infty}\frac{\sin\frac1x\cos x+\cos\frac1x\sin x}{x^p}\,dx
            $$
            其中当 $x\to+\infty$, $\frac{\sin\frac1x}{x^p}$ 和 $\frac{\cos\frac1x}{x^p}$ 都单调趋于 $0$\\
            因此收敛.
        \subsection*{12.}
            $$
                \begin{aligned}
                    \int_{0}^{1}f(x)\,dx\text{ 收敛}&\Rightarrow \forall \varepsilon > 0, \exists \delta > 0, \forall x_0\in (0, \delta): |\int_{\frac{x_0}{2}}^{x_0}f(x)\,dx|<\varepsilon\\
                    &\Rightarrow \frac{x_0}{2}|f(\xi)|<\varepsilon\\
                    &\Rightarrow \xi|f(\xi)|<4\varepsilon\\
                    &\Rightarrow \lim_{x\to 0^{+}}xf(x)=0
                \end{aligned}
            $$
        \subsection*{13.}
            当 $x\to +\infty$ 时, $xf(x)$ 单调减少趋于 $0$, 于是有 $xf(x)\leq 0$
            $$
                0\leq \frac12x(\ln x)f(x)\leq\int_{\sqrt{x}}^{x}tf(t)\frac1t\,dt=\int_{\sqrt{x}}^{x}f(t)\,dt
            $$
            因为 $\int_{a}^{+\infty}f(x)\,dx$ 收敛, 由 Cauchy 收敛原理, $\lim_{x\to+\infty}\int_{\sqrt x}^{x}f(t)\,dt$, 因此
            $$
                \lim_{x\to+\infty}x(\ln x)f(x)=0
            $$
        \subsection*{15.}
            \subsubsection*{(1)}
                $$
                    \int_{a}^{+\infty} f^2(x)\,dx\text{ 收敛无法推出 } \int_{a}^{+\infty} f(x)\,dx \text{ 收敛}
                $$
                反例:
                $$
                    f(x)=\frac1x,a=1
                $$
                $$
                    \int_{a}^{+\infty} f(x)\,dx\text{ 收敛无法推出 } \int_{a}^{+\infty} f^2(x)\,dx \text{ 收敛}
                $$
                反例:
                $$
                    f(x)=\frac{\sin x}{\sqrt{x}},a=1
                $$
            \subsubsection*{(2)}
                $$
                    f(x)=\left\{\begin{aligned}
                        &\frac{1}{\sqrt{\frac{1}{2^{n-1}}-x+n}}, x\in (n,n+\frac{1}{2^n}),n=1,2,...\\
                        &0, \text{其他}
                    \end{aligned}\right.
                $$
                $f(x)$ 是绝对收敛的, 但不是平方可积的.
                $$
                    f(x)=\frac{1}{x^{0.6}},a=1
                $$
                $f(x)$ 是平方可积的, 但不是绝对收敛的.
            \subsubsection*{(3)}
                $|f(x)|\leq \frac12(1+f^2(x))$,因此平方可积必定绝对收敛.
                $$
                    f(x)=\frac{1}{\sqrt{1-x}},a=0,b=1
                $$
                $f(x)$ 是绝对收敛的, 但不是平方可积的.
\end{document}