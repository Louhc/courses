\documentclass[12pt, a4paper, oneside]{ctexart}
\usepackage{amsmath, amsthm, amssymb, appendix, bm, graphicx, hyperref, mathrsfs}

\title{\textbf{Lebesgue定理}}
\author{楼翰诚}
\date{\today}
\linespread{1.5}
\newtheorem{theorem}{定理}[section]
\newtheorem{definition}[theorem]{定义}
\newtheorem{lemma}[theorem]{引理}
\newtheorem{corollary}[theorem]{推论}
\newtheorem{example}[theorem]{例}
\newtheorem{proposition}[theorem]{命题}
\renewcommand{\abstractname}{\Large\textbf{摘要}}

\begin{document}

\maketitle

\setcounter{page}{0}
\maketitle
\thispagestyle{empty}

\begin{abstract}
    这里是摘要. 
    \par\textbf{关键词:}这里是关键词; 这里是关键词. 
\end{abstract}

\newpage
\pagenumbering{Roman}
\setcounter{page}{1}
\tableofcontents
\newpage
\setcounter{page}{1}
\pagenumbering{arabic}

\section{介绍}

\subsection{零测集}

零测集

\newpage

\begin{thebibliography}{99}
    \bibitem{a}作者. \emph{文献}[M]. 地点:出版社,年份.
    \bibitem{b}作者. \emph{文献}[M]. 地点:出版社,年份.
    https://zhuanlan.zhihu.com/p/366160014
\end{thebibliography}