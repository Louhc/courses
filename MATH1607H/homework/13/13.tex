\documentclass[12pt, a4paper, oneside]{ctexart}

\usepackage[top=2cm, bottom=1.5cm, left=1.8cm, right=1.5cm]{geometry} % 页边距
\usepackage{upgreek, graphicx, bm, slashed, amsmath, amssymb, lmodern, simplewick, color, fancyhdr}
%\spaceskip=0.2em % 调节空格大小
\title{Fall 2022 MATH1607H Homework 13}
\author{Lou Hancheng \quad louhancheng@sjtu.edu.cn}
\date{\today}
\pagestyle{fancy}
\fancyhead{}
\fancyhead[L]{MATH1607H}
\fancyhead[R]{Lou Hancheng (522010910160)}

\begin{document}
    \maketitle
    \section*{第9章第1节}
        \subsection*{1.(3)}
            $$
                S_m=\sum_{n=1}^{m}\frac{1}{n(n+1)(n+2)}=\sum_{n=1}^{m}(\frac{1}{2n(n+1)}-\frac{1}{2(n+1)(n+2)})=\frac14-\frac{1}{2(m+1)(m+2)}
            $$
            因此 $\{S_m\}$ 收敛.
            $$
                S=\lim_{m\to+\infty}S_m=\frac14
            $$
        \subsection*{1.(5)}
            $$
                \forall n \in N_+:2^n\geq n,2\geq \sqrt[n]{n}
            $$
            因此
            $$
                \frac{1}{\sqrt[n]{n}}\geq \frac12,S_m\geq\frac{m}{2}
            $$
            所以不收敛.
        \subsection*{1.(7)}
            $$
                S_m=1-\sqrt2+\sqrt{m+2}-\sqrt{m+1}
            $$
            因此 $\{S_m\}$ 收敛.
            $$
                S=\lim_{m\to+\infty}S_m=1-\sqrt2
            $$
        \subsection*{1.(9)}
        \subsection*{2.(2)}
            当 $x=0$ 时, $S_m=m$ 显然发散.\\
            当 $x<0$ 时,
            $$
                S_m=\frac{1-e^{x(m+1)}}{1-e^x}\to \frac{1}{1-e^x}(m\to+\infty)
            $$
            当 $x>0$ 时,
            $$
                S_m=\frac{e^{x(m+1)}-1}{e^x-1}
            $$
            发散.\\
            综上, 当 $x<0$ 时收敛.
        \subsection*{2.(3)}
            $$
                S_m=x-x^{m+1}
            $$
            当 $-1\leq x\leq 1$ 时收敛.
        \subsection*{3.}
            $$
                (10000)_8x-x=(360700)_8
            $$
            $$
                x=\frac{(360700)_8}{(7777)_8}=\frac{123328}{4035}
            $$
    \section*{第1章第2节}
        \subsection*{1.(3)}
        \subsection*{1.(4)}
        \subsection*{1.(5)}
        \subsection*{3.}
\end{document}