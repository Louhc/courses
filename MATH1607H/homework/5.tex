\documentclass[12pt, a4paper, oneside]{ctexart}

\usepackage[top=2cm, bottom=1.5cm, left=1.8cm, right=1.5cm]{geometry} % 页边距
\usepackage{upgreek, graphicx, bm, slashed, amsmath, amssymb, lmodern, simplewick, color, fancyhdr}
%\spaceskip=0.2em % 调节空格大小
\title{Fall 2022 MATH1607H Homework 5}
\author{Lou Hancheng \quad louhancheng@sjtu.edu.cn}
\date{\today}
\pagestyle{fancy}
\fancyhead{}
\fancyhead[L]{MATH1607H}
\fancyhead[R]{Lou Hancheng (522010910160)}

\begin{document}
    \maketitle
    \thispagestyle{headings}
    \textbf{第4章第2节}\\
    1.(3) $\lim\limits_{h\to 0}\frac{f(x_0+h)-f(x_0-h)}{h}=\lim\limits_{h\to0}\frac{f(x_0+h)-f(x_0)+f(x_0)-f(x_0-h)}{h}=\lim\limits_{h\to0}\frac{f(x_0+h)-f(x_0)}{h}+\lim\limits_{h\to0}\frac{f(x_0+(-h))-f(x_0)}{-h}=2f'(x_0)$\\
    4. 椭圆上半部分可以改写为 $y=\frac ba\sqrt{a^2-x^2}$,
    则它在$(x_0,y_0)$处的切线为$y=(tan\theta_0)x+\sqrt{a^2(\tan\theta_0)^2+b^2} (\tan\theta_0=y'(x_0)=-\frac{bx_0}{a\sqrt{a^2-x^2}})$.
    设$(x_0,y_0)$与左焦点的连线与x轴的夹角为$\theta_1$,
    那么$\tan\theta_1=\frac{y_0}{x_0+c}=\frac{b\sqrt{a^2-x_0^2}}{a(x_0+c)}$.
    同理,对于$(x_0,y_0)$与右焦点连线与x轴的夹角为$\theta_2$,
    $\tan\theta_2=\frac{y_0}{x_0-c}=\frac{b\sqrt{a^2-x_0^2}}{a(x_0-c)}$.
    $\tan2\theta_0=\frac{2\tan\theta_0}{1-\tan^2\theta_0}=\frac{2abx_0\sqrt{a^2-x_0^2}}{(a^2+b^2)x_0^2-a^4}$,
    而$\tan(\theta_1+\theta_2)=\frac{\tan\theta_1+\tan\theta_2}{1-\tan\theta_1\tan\theta_2}=\frac{2abx_0\sqrt{a^2-x_0^2}}{(a^2+b^2)x_0^2-a^4}$.
    显然$\tan2\theta_0=\tan\theta_1+\tan\theta_2$,$2\theta_0-(\theta_1+\theta_2)=k\pi$,得证.\\
    5. $y=\frac{a^2}{x}$,$y'(x)=-\frac{a^2}{x^2}$,则其在$(x_0,y_0)$处的切线为$y=-\frac{a^2}{x_0^2}x+2\frac{a^2}{x_0}$,它在x轴和y轴的截距分别为$2x_0$,$2\frac{a^2}{x_0}$,则$S_{\Delta}=2a^2$\\
    6.(2) $y'_-(2k\pi)=\lim\limits_{\Delta x\to0^-}\frac{\sqrt{1-\cos(2k\pi+\Delta x)}-\sqrt{1-\cos 2k\pi}}{\Delta x}
    =\lim\limits_{\Delta x\to0^-}\frac{\sqrt{1-\cos(2k\pi+\Delta x)}}{\Delta x}
    =\lim\limits_{\Delta x\to0^-}\frac{\sqrt{2}|\sin(k\pi+\frac{\Delta x}{2})|}{\Delta x}=-\frac{\sqrt2}{2}$
    同理,$y'_+(2k\pi)=\frac{\sqrt2}{2}$.\\
    6.(4) $y'_-(0)=\lim\limits_{\Delta x\to 0^-}\frac{-\ln(1+\Delta x)}{\Delta x}=-1$,
    同理$y'_+(0)=1$\\
    7.(3) $y'_-(0)=\lim\limits_{\Delta x\to0^-}\frac{\Delta xe^{\Delta x}}{\Delta x}=1$,
    $y'_+(0)=\lim\limits_{\Delta x\to0^+}\frac{a\Delta x^2}{\Delta x}=0$,因此在$x=0$不可导.\\
    7.(4) $y'_-(0)=\lim\limits_{\Delta x\to0^-}\frac{e^{\frac{a}{\Delta x^2}}}{\Delta x}$.若$a>0$,显然左导数不存在.
    若$x=0$,$y'_-(0)=y'_+(\Delta)=0$,因此在$x=0$处可导.
    若$x<0$,$y'_-(0)=\lim\limits_{\Delta x\to0^-}\frac{e^{\frac{a}{\Delta x^2}}}{\Delta x}=\lim\limits_{\Delta x\to 0^-}e^{\frac{a}{\Delta x^2}-\Delta x+1}=0$,
    同理$y'_+(0)=0$.综上,当$a\leq 0$时y(x)可导.\\
    8. 若$f(0)=0$,则$\lim\limits_{\Delta x\to0}\frac{|f(\Delta x)|}{\Delta x}$存在当且仅当
    $\lim\limits_{\Delta x\to0}\frac{|f(\Delta x)|}{\Delta x}=\lim\limits_{\Delta x\to0^+}\frac{|f(\Delta x)|}{\Delta x}=\lim\limits_{\Delta x\to0^-}\frac{|f(\Delta x)|}{\Delta x}=0$,
    即当且仅当$f'(x)=0$时$|f(x)|$在$x=0$处可导.
    若$f(0)\neq 0$,那么存在一个邻域使得$\forall x\in B_{\delta}(0)\backslash\{0\},sgn(f(x))=sgn(f(0))$,
    则$\lim\limits_{\Delta x\to0}\frac{|f(\Delta x)|-|f(0)|}{\Delta x}=\lim\limits_{\Delta x\to0}\frac{|f(\Delta x)-f(0)|}{\Delta x}=|f'(0)|$,此时|f(x)|在$x=0$处可导.\\
    9. 不妨设$f'_+(a)>0,f'_-(b)>0$,$\lim\limits_{x\to a^+}\frac{f(x)-f(a)}{x-a}=f'_+(a)>0$,因此$\exists x_1:\frac{f(x_1)-f(a)}{x_1-a}>0$,即$f(x_1)>0$,
    同理$\exists x_1:f(x_1)<0$,由零点存在定理得,$f(x)$在$(a,b)$至少存在一个零点.\\
    10.(1) 不一定.$f(x)=\frac1x+\cos \frac1x$,$a=0$,显然$f(x)\to\infty(x\to 0^+)$,但是$f'(x)(x\to 0^+)$显然极限不存在.\\
    10.(2) 不一定.$f(x)=\sqrt{x}$,$a=0$,$f'(x)=\frac{1}{2\sqrt{x}}$\\
    11. 充分性: $f(x)=xg(x)$,$\lim\limits_{\Delta x\to 0}\frac{\Delta xg(\Delta x)}{\Delta x}=g(0)$,因此$f(x)$在$x=0$上可导且$f'(0)=g(0)$\\
    必要性: 设$g(x)=\left\{\begin{aligned}&\frac{f(x)}{x},x\neq 0\\&f'(x),x=0\end{aligned}\right.$, $\lim\limits_{\Delta x}$,那么$f(x)=xg(x)$且$g(0)=f'(0)$,
    $\lim\limits_{\Delta x\to 0}g(\Delta x)=\lim\limits_{\Delta x\to 0}\frac{f(\Delta x)}{\Delta x}=f'(0)=g(0)$,因此$g$在$x=0$连续.\\
    
    \textbf{第4章第3节}\\
    2.(3) $(\arccos x)'=\frac{1}{\cos'(\arccos x)}=-\frac{1}{\sin(\arccos x)}=-\frac{1}{\sqrt{1-x^2}}$\\
    2.(6) $(th^{-1}x)'=\frac{1}{th'(th^{-1}x)}=\frac{1}{1-th^2(th^{-1}x)}=\frac{1}{1-x^2}$,
    $(cth^{-1}x)'=\frac{1}{cth'(cth^{-1}x)}=\frac{1}{1-cth^2(cth^{-1}x)}=\frac{1}{1-x^2}$\\
    3.(2) $f'(x)=\cos x-x\sin x+2x$\\
    3.(6) $f'(x)=\frac{(2\cos x+1-2^x\ln x)x^{\frac23}-\frac23(2\sin x+x-2^x)x^{-\frac13}}{x^{\frac43}}$\\
    3.(9) $f'(x)=\frac{(3x^2-\csc^2x)\ln x-x^2-\frac{\cot x}{x}}{\ln^2x}$\\
    3.(11) $f'(x)=(e^x+\frac{1}{x\ln3})\arcsin x+(e^x+\log_3x)\frac{1}{\sqrt{1-x^2}}$\\
    3.(14) $f'(x)=\frac{(1+\cos x)\arctan x-\frac{x+\sin x}{1+x^2}}{\arctan^2x}$\\
    5. 设相切于$P(x_0,y_0)$, 由于$P$在直线$y=x$上和曲线$y=\log_ax$上,因此$\left\{\begin{aligned}&y_0=x_0\\&y_0=\log_ax_0\end{aligned}\right.$
    由于直线$y=x$上和曲线$y=\log_ax$在$P$上相切, 因此在$P$点导数相等,即 $1=\frac{1}{x_0\ln a}$.解得$\left\{\begin{aligned}&x_0=e\\&y_0=e\\&a=e^{\frac1e}\end{aligned}\right.$,
    切点为$(e,e)$.\\
    8.(1) 反证法.假设$c_1f(x)+c_2g(x)$在$x=x_0$处可导,又因为$f(x)$在$x=x_0$处可导,
    那么$g(x)=\frac{(c_1f(x)+c_2g(x))-c_1f(x)}{c_2}$在$x=x_0$处可导,与$g(x)$在$x=x_0$处不可导矛盾,
    因此$c_1f(x)+c_2g(x)$在$x=x_0$处不可导.\\
    8.(2) 不能.如果$f(x)=g(x)=|x|$,则$f+g$在$x=x_0$处不可导,$f-g$在$x=x_0$处可导.\\
    9. $f(x)=x$,$g(x)=|x|$时$f(x)g(x)$在$x=0$处可导,$f(x)=1$,$g(x)=|x|$时$f(x)g(x)$在$x=0$处不可导.
    $f(x)=g(x)=|x|$时$f(x)g(x)$在$x=0$处可导,$f(x)=|x|+|x+1|,g(x)=|x|$时$f(x)g(x)$在$x=0$处不可导.\\
    10. QAQ\\
    
    \textbf{第4章第4节}\\
    1.(4) $y'=\frac{1-\ln x}{2x^2}\sqrt{\frac{x}{\ln x}}$\\
    1.(8) $y'=-\frac{2xe^{-x^2}}{\sqrt{1-e^{-2x^2}}}$\\
    1.(9) $y'=(2x+\frac{2}{x^3})\frac{1}{x^2-\frac{1}{x^2}}=\frac{2x^4+2}{x^5-x}$\\
    1.(12) $y'=\frac{1+\csc x^2+x^2\cot x^2\csc x^2}{(1+\csc x^2)^{\frac32}}$\\
    1.(14) $y'=(\cos x)(-2\sin x)e^{-\sin^2x}=-2\sin x\cos xe^{-\sin^2x}$\\
    2.(4) $y'=\frac{1+\frac{x}{\sqrt{x^2+a^2}}}{x+\sqrt{x^2+a^2}}=\frac{1}{\sqrt{x^2+a^2}}$\\
    2.(5) $y'=\frac12\sqrt{x^2-a^2}+\frac{x^2}{2\sqrt{x^2-a^2}}-\frac{a^2}{2}(1+\frac{x}{\sqrt{x^2-a^2}})\frac{1}{x+\sqrt{x^2-a^2}}
    =\sqrt{x^2-a^2}$\\
    3.(5) $[f(f(e^{x^2}))]'=2xe^{x^2}f'(e^{x^2})f'(f(e^{x^2}))$\\
    3.(7) $[f(\frac{1}{f(x)})]'=f'(x)(-\frac{1}{f^2(x)})f'(\frac{1}{f(x)})=-\frac{f'(x)f'(\frac{1}{f(x)})}{f^2(x)}$\\
    3.(8) $[\frac{1}{f(f(x))}]'=f'(x)f'(f(x))(-\frac{1}{f^2(f(x))})=-\frac{f'(x)f'(f(x))}{f^2(f(x))}$\\
    4.(3) $\ln y=x\ln\cos x$,两边求导得,$\frac{y'}{y}=\ln\cos x-x\tan x$,$y'=(\ln\cos x-x\tan x)\cos^xx$\\
    4.(5) $\ln y=\ln x+\ln(1-x^2)+\ln(1+x^3)$,两边求导得,$\frac{y'}{y}=\frac{1}{x}+\frac{-2x}{1-x^2}+\frac{3x^2}{1+x^3},
    y'=(1-\frac{2x^2}{1-x^2}+\frac{3x^3}{1+x^3})\sqrt{\frac{1-x^2}{1+x^3}}$\\
    4.(7) $\ln \arcsin y=\sqrt x\ln x$,两边求导得,$\frac{y'}{\sqrt{1-y^2}\arcsin y}=\frac{1}{\sqrt x}+\frac{\ln x}{2\sqrt x}$,
    $y'=\frac{2+\ln x}{2\sqrt x}x^{\sqrt x}\cos x^{\sqrt x}$\\
    6.(1) 设$f$是偶函数,$f(x)=f(-x)$.$f'(-x)=\lim\limits_{\Delta x\to 0}\frac{f(-x+\Delta x)-f(-x)}{\Delta x}
    =-\lim\limits_{\Delta x\to 0}\frac{f(x+(-\Delta x))-f(x)}{(-\Delta x)}=-f'(x)$,因此$f'$是奇函数.
    设$g$是奇函数,同理可得$g'$是偶函数.\\
    6.(2) 设$h$是周期为$T$的周期函数,$h(x+T)=h(x)$.$h'(x+T)=\lim\limits_{\Delta x\to 0}\frac{h(x+T+\Delta x)-h(x+T)}{\Delta x}
    =\lim\limits_{\Delta x\to 0}\frac{h(x+\Delta x)-h(x)}{\Delta x}=h'(x)$.因此$h'$是周期函数.\\
    12.(1) $g(x)=x^2,f(u)=|u|,x_0=u_0=0$\\
    12.(2) $g(x)=|x|,f(u)=u^2,x_0=u_0=0$\\
    12.(3) $g(x)=|x|,f(u)=\left\{\begin{aligned}&x^2,x\geq 0\\&x,x<0\end{aligned}\right.,x_0=u_0=0$\\
    13.(1) $d[f(u)g(u)h(u)]=(f'(u)g(u)h(u)+f(u)g'(u)h(u)+f(u)g(u)h'(u))du=(f'(u)g(u)h(u)+f(u)g'(u)h(u)+f(u)g(u)h'(u)\varphi'(x)dx$ \\
    13.(4) $d[\log_{h(u)}g(u)]=\frac{g'(u)h(u)\ln h(u)-h'(u)g(u)\ln g(u)}{g(u)h(u)\ln^2h(u)}du=\frac{g'(u)h(u)\ln h(u)-h'(u)g(u)\ln g(u)}{g(u)h(u)\ln^2h(u)}\varphi'(x)dx$\\
    13.(6) $d[\frac{1}{\sqrt{f^2(u)+h^2(u)}}]=-\frac{f(u)f'(u)+h(u)h'(u)}{(f^2(u)+h^2(u))^{\frac32}}du=-\frac{f(u)f'(u)+h(u)h'(u)}{(f^2(u)+h^2(u))^{\frac32}}\varphi'(x)dx$\\
    
    \textbf{第4章第4节}\\
    5.(6) $\tan(x+y)-xy=0$, $\frac{1+y'}{\cos^2(x+y)}-y-xy'=0$, $y'=\frac{1-y\cos^2(x+y)}{x\cos^2(x+y)-1}$\\
    5.(8) $x^3+y^3-3axy=0$, $3x^2+3y^2y'-3ay-3axy'=0$, $y'=\frac{3x^2-3ay}{3ax-3y^2}$.\\
    7. $xy+\ln y=1$, $y+xy'+\frac{y'}y=0$, $y'=-\frac{y^2}{xy+1}=-\frac12$,因此切线为$y=-\frac12x+\frac32$, 法线为$y=2x-1$\\
    9. $\frac{dx}{dt}=\frac{(2+2t)(1+t^3)-3t^2(2t+t^2)}{(1+t^3)^2}$, $\frac{dy}{dt}=\frac{(2-2t)(1+t^3)-3t^2(2t-t^2)}{(1+t^3)^2}$, $\frac{dy}{dx}=\frac{(2-2t)(1+t^3)-3t^2(2t-t^2)}{(2+2t)(1+t^3)-3t^2(2t+t^2)}=3$, 切线 $y=3x-4$, 法线 $y=-\frac x3+1$\\
    11. $\frac{dx}{dt}=a(-\sin t+\sin t+t\cos t)=at\cos t,\frac{dy}{dt}=a(\cos t-\cos t+t\sin t)=at\sin t, \frac{dy}{dx}=\tan t$,在$t=t_0$处法线为$y=(-\cot t_0)x+a\cot t_0(\cos t_0+t_0\sin t_0)+a(\sin t_0-t_0\cos t_0)$
    它到原点的距离为$|\frac{a\cot t_0(\cos t_0+t_0\sin t_0)+a(\sin t_0-t_0\cos t_0)}{\sqrt{\cot^2t_0+1}}|=|\frac{\frac{a}{\sin t_0}}{\frac1{\sin t_0}}|=a$\\
    
    \textbf{第4章第5节}\\
    1.(4) $y'=\frac{\frac1xx^2-2x\ln x}{x^4}=\frac{1-2\ln x}{x^3}$, $y''=\frac{-\frac2xx^3-3x^2(1-2\ln x)}{x^6}=\frac{6\ln x-5}{x^4}$\\
    1.(9) $y^{(80)}=x^3(\cos 2x)^{(80)}+80x^2(\cos 2x)^{(79)}+3160x(\cos 2x)^{(78)}+82160(\cos 2x)^{(77)}=2^{80}x^3\cos 2x+80\cdot2^{79}x^2\sin 2x-3160\cdot2^{78}x\cos 2x-82160\cdot2^{77}\sin 2x$\\
    2.(5) $y^{(n)}=\sum\limits_{i=0}^n\binom{n}{i}a^i\beta^{n-i}e^{ax}\cos(\beta x+\frac i2\pi)$\\
    3. $f'(x)=\left\{\begin{aligned}& 2x, x\geq 0\\& -2x, x<0\end{aligned}\right.$, $f''(x)=\left\{\begin{aligned}&2,x>0\\&doesn't~exist,x=0\\&-2,x<0\end{aligned}\right.$, $\forall n \geq 3,f^{(n)}(x)=\left\{\begin{aligned}&0,x>0~or~x<0\\&doesn't~exist,x=0\end{aligned}\right.$\\
    4.(5) $[f(e^{-x})]'''=[-e^{-x}f'(e^{-x})]''=[e^{-2x}f''(e^{-x})+e^{-x}f'(e^{-x})]'=-3e^{-2x}f''(e^{-x})-e^{-3x}f'''(e^{-x})-e^{-x}f'(e^{-x})$\\
    5.(1) $y'=\frac{1}{x^2+1}$, $(x^2+1)y'=1$, 两边求$n-1$阶导得 $\sum\limits_{i=0}^{n-1}\binom{n-1}{i}(x^2+1)^{(n-1-i)}(y')^{(i)}=0$, 因此 $y^{(n)}(0)=-2\binom{n-1}{2}y^{(n-2)}(0)=-(n-1)(n-2)y^{(n-2)}(0)$.$y''=\frac{2x}{(x^2+1)^2}$, $y''(0)=0$. So $y^{(n)}=\left\{\begin{aligned}&0,n=2k\\&(-1)^{k-1}(n-1)!,n=2k-1\end{aligned}\right.$.\\
    6.(2) $\frac{1+y'}{\cos^2(x+y)}-y-xy'=0$, $\frac{d^2y}{dx^2}=\frac{y\cos^2(x+y)-1}{1-x\cos^2(x+y)}$, $\frac{y''+2(1+y')^2\tan(x+y)}{\cos^2x}-2y'-xy''=0$, $y''=\frac{2(1+y')^2\tan(x+y)-2y'\cos^2x}{x\cos^2x-1}$\\
    7.(3) $\frac{dx}{dt}=1-\sin t-t\cos t$, $\frac{dy}{dt}=\cos t-t\sin t$, $\frac{dy}{dx}=\frac{\cos t-t\sin t}{1-\sin t-t\cos t}$, \\$\frac{d^2y}{dxdt}=\frac{(-2\sin t-t\cos t)(1-\sin t-t\cos t)-(\cos t-t\sin t)(-2\cos t+t\sin t)}{(1-\sin t-t\cos t)^2}$, $\frac{d^2y}{dx^2}=\frac{t^2+2-2\sin t-t\cos t}{(1-\sin t-t\cos t)^3}$\\
    8.(2) $\frac{d^2x}{dy^2}=\frac{d(\frac{dx}{dy})}{dx}=\frac{-\frac{1}{y'(x)}y''(x)}{(y'(y^{-1}(y)))^2}=-\frac{y''}{(y')^3}$, $\frac{d^3x}{dy^3}=\frac{d(\frac{d^2x}{dy^2})}{dy}=-\frac{1}{y'}\cdot\frac{y'''(y')^3-3(y''y')^2}{(y')^6}=\frac{3(y'')^2-y'''y'}{(y')^5}$\\
    12. 当$n=0$, $\text{左}=x^{-1}e$, $\text{右}=\frac1xe^{\frac1x}=\text{左}$.\\
    设当$n\leq k$时,$(x^{n-1}e^{\frac1x})^{(n)}=\frac{(-1)^n}{x^{n+1}}e^{\frac1x}$.\\
    当$n=k+1$时,$(x^{n-1}e^{\frac1x})^{(n)}=(x^{k}e^{\frac1x})^{(k+1)}=k(x^{k-1}e^{\frac1x})^{(k)}-(x^{k-2}e^{\frac1x})^{(k)}=k\frac{(-1)^k}{x^{k+1}}e^{\frac1x}-(\frac{(-1)^{k-1}}{x^{k}}e^{\frac1x})'=k\frac{(-1)^k}{x^{k+1}}e^{\frac1x}-k\frac{(-1)^k}{x^{k+1}}e^{\frac1x}=0$.\\
    因此$(x^{n-1}e^{\frac1x})^{(n)}=\frac{(-1)^n}{x^{n+1}}e^{\frac1x}$.\\
    
    \textbf{第5章第1节}\\
    1. $\lim\limits_{x\to x_0^+}\frac{f(x)-f(x_0)}{x-x_0}>0$, 因此$\exists \delta_1>0,\forall x\in(x_0,x_0+\delta_1):f(x)>f(x_0)$.同理,$\exists \delta_2>0,\forall x\in(x_0-\delta_2,x_0):f(x)>f(x_0)$,因此$x_0$是$f$的极小值点\\
    4. $\psi(x)=\left|\begin{matrix}x&f(x)&1\\a&f(a)&1\\b&f(b)&1\end{matrix}\right|=(a-b)(\frac{f(a)-f(b)}{a-b}x+\frac{af(b)-bf(a)}{a-b}-f(x))$.\\
    $\psi(a)=\psi(b)=0$, 设$M,m$分别为$\psi(x)$在$[a,b]$上的最大值和最小值.如果$m=M$,那么$\psi$为常值函数,$\forall x\in(a,b):\psi'(x)=0$.若$M\neq m$,则$M$与$m$至少有一个不为$0$,其为$\psi$的极值.因为存在极值点,由Fermat引理,$\exists x\in(a,b):\psi'(x)=0$\\
    综上,$\exists x\in (a,b):\psi'(x)=0$, 即$\exists x\in (a,b):(a-b)(\frac{f(a)-f(b)}{a-b}-f'(x))=0,\frac{f(a)-f(b)}{a-b}=f'(x)$.得证.\\
    $\psi$的几何含义为$f(x)$到端点连线的垂直距离的$(b-a)$倍(在上方为正,在下方为负),\\
    即$(a,f(a)),(b,f(b)),(x,f(x))$三点构成的三角形面积的两倍($(x,f(x))$在$(a,f(a)),(b,f(b))$连线上方值为正,否则为负).
\end{document}